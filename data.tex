We examine the phase delay effect bin
212 software projects from around the world from 
\bill{2005 to 2015} that used the Team Software Process (TSP). 

\subsection{Overview of the Team Software Process}
The Software Engineering Institute (SEI) at
Carnegie Mellon University explores methods for software process improvement.
One product from that research is  
Watts Humphrey's  Team Software ProcessSM (TSPSM)~\cite{tsp00}. TSP is an extension of Humphrey's early
work on Personal Software
ProcessSM (PSPSM)~\cite{psp05}.



\sei{One paragraph summary on the key points of TSP execution? The following text should also be reviewed and better integrated.}

TSP guiding principles include 1) engineering work does not happen by accident, it must be planned, 2) the plans must include realistic commitments of the resources needed to perform the work, 3) planed and actual results should be frequently examined to assure that plans remain relevant, and 4) people performing the work should be responsible for planning and tracking the work. After clarifying goals, defining work practices, and estimating size and effort, developers make a detailed short term plan. As the developers perform project work, they use a tool such as the Process Dashboard to collect their time, effort, size, and schedule data. Every week, the team review their data to evaluate status, review actual rates, determine if project goals for schedule, cost, and quality are being met and then make necessary plan corrections. At the end of the project the coach and team perform a quantitative post mortem.

Four common features of TSP projects include (1)~coaches,
(2)~peer review, and  (3)~personnel reviews (4)~planning.
 
A ``coach'' works with the team and is authorized to submit project data.
Before reviewing data with the teams, therefore before submission,
these coaches check the data for obvious errors.

``Personnel review'' is a technique taken  from the Personal Software
ProcessSM (PSPSM) and is more or less unique in its TSP implementation. The technique of a personal review is similar to that of a peer review. Developers systematically examine their own work products using a checklist built from their personal defect profile. This personal review occurs after a portion of a product is done and before peer reviews. 

%PSPSM teaches developers howto continually make and review their personnel estimates
%about their day-to-day tasks, then compare those estimates against the actual development effort.
%In this way, developers can acquire a more realistic understanding of their work behaviour.
  
 
``Peer review'' is a  technique in
traditional software engineering.
 Basili and Boehm write  commented in 2001~\cite{boehm01} 
that peer reviews can catch over half the defects introduced into a system.
Peer review can be conducted on any artifact generated anywhere in the software
lifecycle and can quickly be adapted to new kinds of artifacts. TSP peer reviews follow the Fagan style in which the reviewer uses a checklist composed of common team defects prior to a review team meeting. 


''Planning '' is another technique from the in the PSP(SM). Developers estimate the size of work products, define the process phases, use historic rates and time in phase to estimate the required time, the number of defects injected, and the defects to be removed in each phase. Coaches help the developers to compare estimates against actual results. In this way, developers acquire a more realistic understanding of their work behavior and performance.



\subsection{Projects in the Sample}
Since 2004, one of the authors (Nichols) has been been mentoring software development teams and coaches around the world as they deploy TSP within their organizations.  
The projects
were mostly small to medium, with a median duration of 46 days
and a maximum duration of 90 days. Median team size was 7
people, with a maximum of 40. Hence,  total development 
effort\footnote{
Effort =  duration*teamSize/workDaysPerYear.} for
these projects ranged from 1.3 years (median) to 14.3 years (max).
The majority of
the projects were either web portals or banking systems in the US, South Africa, and Mexico. 
There were also some  medical devices (US, France, Japan, and Germany),  a few from a commercial 
computer-aided design system (developed by a world wide distributed team.
 An anonymized version of that data is available in the PROMISE repository\footnote{\carter{Need to add}} and,
 for confidentiality restrictions, we cannot offer 
further details
on those project.  

\bill{can we offer any more deatils here?}
  
  

\subsection{Data Collection Process}



Organizations using a TSP agree to provide their project data to the SEI for use in research. In return the SEI agrees    that  data must not be traceable to its source. The data is collected at major project events, launch, interim checkpoints, and at completion. Data includes project and site  characteristic summaries, surveys of team members, launch presentations, launch outbriefs, and baseline plans, final data from the project, and the post mortem report.  In practice, this data requirement has only been enforced for the purposes of certifying and reauthorizing TSP coaches who are the only individuals authorized to submit data. To this data, only the data recorded using the Process Dashboard tool has been collected and aggregated for research. 

To aid in using this data for research and benchmarking  Nichols first collected the Process Dashboard data from the submission repository then used a tool built by the Process Dashboard developer, David Tuma, 
to gather the project data into a database at SEI. Nichols and Shirai then extracted data into views suitable for analysis. . The key views include the a project results summary, effort logs, task logs, and defect logs. At this stage, the data has not yet undergone additional screening. 

\sei{
do we need the following information? we do not use it in this analysis
}

The issue types collected in the the SEI TSP data set divide into:
\be 
\item Environment: design, compile, test, or other support system problems;
\item Interface: procedure calls and reference, I/O, user format;
\item Data: structure, content; 
\item Documentation: comments, messages;
\item Syntax: spelling, punctuation typos, instruction formats
\item Function: logic, pointers, loops, recursion, computation, function defects  
\item Checking: error messages, inadequate checks;
\item Build: change management, library, version control;
\item Assignment: package
declaration, duplicate names, scope, limits;
\item System: configuration, timing, memory.
\ee
The tool also collected process information about those defects.
That data includes  work start time, work end time, delta
work time, and interruption time. Software engineers are often
interrupted by meetings, requests for technical help, reporting, and
so forth. These events are recorded, in minutes, as interruption
time. In this paper, when we report ``time to resolve an
issue'', we show the difference between the start and end times
of a work session, with any interruption time subtracted (the
difference in times, minus the interruptions).  

%wrn, updated 3/11 these
As of February 2015, the SEI TSP database contained data from 212
TSP projects. The projects completed between July 2008 and
November 2014; they included 47 organizations and 843 people. Among
the database fact tables, 
contains 268,726 time logs, 
154,238 task logs,
 47,376 defect logs, 
and 26,534 size logs
%\footnote{\bill{we need toupdate this table}}.  


%That said, certain semantic features of the SEI TSP data should be noted.
%Firstly, in the current TSP collection tool, 
%fix times are only the developer time for the developer walking through the phases of \fig{waterfall}.
%We are currently tracking the fix time for post-release issues (e.g. those raised during  acceptance test and %later
%product life cycle). So far, in that post-release data,  we have not detected
%a dramatic phase escalation effect (but at this time, we have nothing definitive comment on that matter).%
%
%\bill{somewhere you have one note on \underline{find} and fix times.  for this paper, we need just fix times. %but is there
%anything we need to fret about re \underline{find} times?}
 
\subsection{Project Phases}
This paper studies the impact of phase delay on the time required to resolve issues.
The logical phases used in this paper are shown in \fig{waterfall}. Note that, in that figure:
\bi 
\item
Several  phases in which product is created have sub-phases of {\em review} and {\em inspect} to remove defects. TSP uses review when individuals perform personal reviews of their work products prior to the peer review which TSP calls the inspection).
\item Testing is divided into several stages. Developers perform unit test prior to code complete. Some standard test phases after code complete include the integration which combines program units into workable system ready for system test. Integration and system test are often performed by a separate group.

\ei 

\todo{I recommend deleting the rest of this section. Replace with better descriptions of what the phases are.
wrn: I agree, I will do this Friday}
Also shown in \fig{waterfall} (bottom left) is the standard definition of an agile process~\cite{boehmturner03}:
Given some
some backlog of tasks, wgeb teams complete their current tasks, they select the next task(s) to complete. That selection process may use a variety of criteria
to prioritize which  tasks are selected (for more details on that selection process, see~\cite{me09j,port08,boehmturner03}). Tasks are completed in ``sprints'' that can last hours,
days, but rarely not more than weeks. Each day meet for brief ``scrum'' sessions to assess (and possibly alter) their current progress on the goals of the sprint.  
Agile teams race to generate releases
(in the continuous release model, releases can be generated on a daily basis, or even faster).  
Experience gained from those releases informs the discussion in the daily scrums which, in turn,
can inform the team's decisions on how to select and implement the next set of tasks for next sprint.
In fact, that experience can result in changing some/all of the tasks in the current backlog. 

TSP is not antithetical to agile--  indeed a TSP-waterfall style project can adopt aspects
of agile.  For example the agile loop  could be applied
over one or more of the phases shown in the long waterfall chain of \fig{waterfall}. Some
TSP teams adopt something like test-driven-development\footnote{TDD is an agile-method
where the ``test'' is the primary driver of the design. The tests are written first,
then the code to support those tests. TDD proceeds in three steps: red (where there
are broken tests); green (where tests are passing); and, possibly, refactor (where
the code is re-organized based on feedback from those tests~\cite{fraser03}.} where
code inspections are scheduled after unit testing. Some of the groups in the SEI TSP data used that approach but
they do not effect the main conclusions of this paper:
\bi 
\item They were less than 5\% of the total \bill{any numbers on this?};

\item Most of our phase delay data comes from much early in the waterfall model
of \fig{waterfall}.
\ei 
That said, there are some very ``un-agile'' aspects
of the processes used by the   projects in the TSP SEI data.
A TSP participant spends much time reflecting on the project,
undisturbed by the behaviour of the  executables.
In fact,
it can be days/weeks
before TSP participants gain  feedback from executing code, for the following reasons:  
\bi 
\item
For TSP projects, the times spent in the of a design activity/phase is   
approximately as much effort as the coding phase.  
\item Our TSP teams did not
 combine development and  testing. 
 \item  The TSP projects studied here typically offer new releases at least every three months, but
often much longer that that\bill{any numbers on that}. Note that this is a far slower
release schedule than seen in, say, continuous integration projects.  
\item The time devoted to personnel review'' and peer review (defined above)
is  about 50\% as much effort as the previous construction phase activity.
 
\ei

Having presented all that, we can now present the main point of this section:
\bi 
\item The results section of this paper fails to find that  phase delay causing dramatic increases
in time to fix issues;
\item This lack-of-phase-delay effect {\em cannot} be explained away just by  saying that
the projects in this sample adopted something like  agile methods to reduce re-work costs.
\ei  